% The formatting of this CV is based on @davidwhogg's layout.

\documentclass[12pt,letterpaper]{article}

\usepackage{color}
\usepackage{fancyhdr}
\usepackage{hyperref}
\usepackage{ifthen}

% Date formatting.
\usepackage[yyyymmdd]{datetime}
\renewcommand{\dateseparator}{-}

% Link formatting.
\definecolor{linkcolor}{rgb}{0,0,0.4}
\hypersetup{%
    colorlinks=true,        % false: boxed links; true: colored links
    linkcolor=linkcolor,    % color of internal links
    citecolor=linkcolor,    % color of links to bibliography
    filecolor=linkcolor,    % color of file links
    urlcolor=linkcolor      % color of external links
}

% Text formatting.
\newcommand{\foreign}[1]{\textit{#1}}
\newcommand{\etal}{\foreign{et~al.}}
\newcommand{\project}[1]{\textsl{#1}}
\definecolor{grey}{rgb}{0.5,0.5,0.5}
\newcommand{\deemph}[1]{\textcolor{grey}{\footnotesize{#1}}}

% literature links--use doi if you can
  \newcommand{\doi}[2]{\emph{\href{http://dx.doi.org/#1}{{#2}}}}
  \newcommand{\ads}[2]{\href{http://adsabs.harvard.edu/abs/#1}{{#2}}}
  \newcommand{\isbn}[1]{{\footnotesize(\textsc{isbn:}{#1})}}
  \newcommand{\arxiv}[1]{{\href{http://arxiv.org/abs/#1}{arXiv:{#1}}}}

% Section headings.
\newcommand{\cvheading}[1]{\addvspace{1ex}\pagebreak[2]\par\textbf{#1}\nopagebreak\vspace{-0.4em}}

% Set up the custom unordered list.
\newcounter{refpubnum}
\newcommand{\cvlist}{%
    \rightmargin=0in
    \leftmargin=0.15in
    \topsep=0ex
    \partopsep=0pt
    \itemsep=0.2ex
    \parsep=0pt
    \itemindent=-1.0\leftmargin
    \listparindent=0.0\leftmargin
    \settowidth{\labelsep}{~}
    \usecounter{refpubnum}
}

% Margins and spaces.
\raggedright
\setlength{\oddsidemargin}{0in}
\setlength{\topmargin}{0in}
\setlength{\headsep}{0.20in}
\setlength{\headheight}{0.25in}
\setlength{\textheight}{9.1in}
\addtolength{\topmargin}{-\headsep}
\addtolength{\topmargin}{-\headheight}
\setlength{\textwidth}{6.50in}
\setlength{\parindent}{0in}
\setlength{\parskip}{1ex}

% Headings and footings.
\renewcommand{\headrulewidth}{0pt}
\pagestyle{fancy}
\lhead{\deemph{Dr Ruth Angus}}
\chead{\deemph{Curriculum Vitae}}
\rhead{\deemph{\thepage}}
\cfoot{\deemph{Last updated: \today}}

% Journal names.
\newcommand{\aj}{AJ}
\newcommand{\apj}{ApJ}
\newcommand{\pasp}{PASP}
\newcommand{\mnras}{MNRAS}


\begin{document}\thispagestyle{empty}\sloppy\sloppypar\raggedbottom

\textbf{\Large Ruth Angus} \hfill
\textsf{\small RuthAngus@gmail.com} \\[0.5ex]
University of Oxford, Denys-Wilkinson Building, Keble Road, Oxford, OX1
3RH\\[0.5ex]

\cvheading{Education}
\begin{list}{}{\cvlist}
\item
D. Phil (Ph.D.) June 2016, Subdepartment of Astrophysics, University of
Oxford.
Advisor: Professor Suzanne Aigrain.
\item
Predoctoral fellowship 2015, Harvard-Smithsonian Center for Astrophysics.
Advisor: Professor John Asher Johnson.
\item
MPhys Physics with Astrophysics 2012, Department of Physics, University of
Southampton, UK. Advisor: Dr David Latham (Harvard-Smithsonian Center for
Astrophysics).
\end{list}

% \cvheading{Academic Awards}
% \begin{list}{}{\cvlist}
% \item
% Predoctoral Fellowship, Harvard-Smithsonian Center for Astrophysics (2014).
% \item
% Highest score in third year undergraduate studies, University of Southampton
% (2011).
% \item
% Highest overall score in undergraduate studies, University of Southampton
% (2012).
% \end{list}

\cvheading{Principal investigator observing plans and observing experience}
\begin{list}{}{\cvlist}
\item
PI: {\it Searching for Super-Earths orbiting evolved stars} Planet Finder
Spectrograph on the Magellan Telescope, Las Campanas, Chile, 2 nights awarded,
2015.
\item
PI: {\it Optimising an observing strategy to search for planets orbiting
evolved stars} Planet Finder Spectrograph on the Magellan Telescope, Las
Campanas, Chile, 2 nights awarded, 2015.
\item
Observer: Sophie spectrograph, Observatoire d'Haute Provence, 6 nights,
September 2013.
\item
Observer: TRES spectrograph, Fred Lawrence Whipple Observatory, 5 nights,
November 2011.
\end{list}

\cvheading{Recent talks and tutorials}
\begin{list}{}{\cvlist}
\item
{\it Exploring gyrochronology with LSST}, Contributed talk, Cool Stars 19,
Uppsala, Sweden, June 2016.
\item
{\it K2 asteroseismology}, Invited review talk, K2 special session, AAS 227,
Kissimee, Florida, January 2016.
\item
{\it Stellar ages from stellar rotation}, Dissertation talk, AAS 227,
Kissimee, Florida, January 2016.
\item
{\it Towards an age for every star: calibrating the age-rotation relations},
Astrophysics Colloquium, The Ohio State University September 2015.
\item
{\it Stellar ages and stellar rotation}, invited seminar, Princeton
University, September 2015.
\item
{\it Probabilistic stellar rotation periods with Gaussian processes},
invited seminar, Institute for Advanced Study, September 2015.
\item
{\it Towards an age for every star: calibrating the age-rotation relations},
invited seminar, Canadian Institute for Theoretical Astrophysics,
University of Toronto, September 2015.
\item
{\it Stellar rotation, asteroseismology and gyrochronology}, invited seminar,
Harvard-Smithsonian Center for Astrophysics, August 2015.
\item
{\it Gyrochronology with K2 and TESS}, contributed talk, International
Astronomical Union meeting XXIX, Honolulu, HI, August 2015.
\item
{\it Probabilistic stellar rotation periods with Gaussian processes},
contributed talk, International Astronomical Union meeting XXIX, Honolulu, HI,
August 2015.
\item
{\it What K2 can do for gyrochronology}, invited seminar, NASA Ames, Mountain
View, CA, July 2015.
\item
{\it An introduction to Gaussian processes}, tutorial, Harvard-Smithsonian
Center for Astrophysics, July 2015.
\item
{\it The Systematics-Insensitive Periodogram for K2}, invited seminar,
Massachussetts Institute for Technology, June 2015.
\item
{\it The Systematics-Insensitive Periodogram for K2}, contributed talk,
Emerging researchers in exoplanets symposium, Penn State University, May 2015.
\item
{\it Gyrochronology and asteroseismology with Kepler and K2}, invited
seminar, Boston University, April 2015.
\item
{\it Calibrating gyrochronology}, contributed talk, American Astronomical
Society meeting, Seattle, WA, January 2015.
\item
{\it Calibrating gyrochronology using Kepler asteroseismic targets}, invited
seminar, University of Sheffield, UK, December 2014.
\item
{\it Calibrating gyrochronology using Kepler asteroseismic targets},
contributed talk, Planets across the HR diagram, Cambridge University, UK,
August 2014.
\item
{\it Calibrating gyrochronology using Kepler asteroseismic targets},
contributed talk, Cool Stars 18, Flagstaff, AZ, June 2014.
{\it Measuring stellar rotation periods using Gaussian processes},
contributed talk, Exoplanet statistics meeting, Carnegie-Mellon University,
Pittburgh, PA, June 2014.
\item
{\it Calibrating gyrochronology using Kepler asteroseismic targets}, invited
seminar, New York University, New York, March 2014.
\item
{\it Gyrochronology, stellar rotation and Gaussian processes}, invited
seminar, Harvard-Smithsonian Center for Astrophysics, February 2014.
\item
{\it Calibrating gyrochronology using Kepler asteroseismic targets}, Kepler
science conference, NASA Ames, Mountain View, CA, November 2013.
\end{list}

\cvheading{Writing, teaching and public outreach}
\begin{list}{}{\cvlist}
\item
Local organiser: dotAstronomy conference, Oxford, UK, June 2016.
\item
Organiser: Cool Stars 19 hack day, Uppsala, Sweden, June 2016.
\item
Scientific expert: Channel 5 News, March 2015.
\item
Scientific expert: The Sky at Night (BBC television), November 2015.
\item
Teaching: Astrophysics labs demonstrator, University of Oxford, 2013-2014.
\item
Writer: BBC focus magazine book reviews (2014).
\item
Writer: Astrobites blog (2013-2015).
\item
{\it The end of the world is nigh!}, Public talk, Green Man festival, Brecon
Beacons, Wales, UK, August 2014.
\item
{\it The end of the world is nigh!}, Public talk, Wilderness festival, Brecon
Beacons, Wales, UK, August 2014.
\item
{\it Exoplanets and where to find them}, Public talk, The Idler magazine,
London, July, 2014.
\item
{\it Dark Sky Status}, Play, {\it Curious Directive} theater company,
Performer and scientific advisor, Latitude festival, UK, July, 2013.
\item
{\it Extrasolar planets: from science fiction to reality}, winner of the
Physics graduate outreach talk competition, University of Oxford, 2012.
\end{list}

\ifdefined\withpubs
    \cvheading{Publications}
    \begin{list}{}{\cvlist}
    \item
{\bf Angus, R.}, Foreman-Mackey, D., \& Johnson, J. A. 2015,
{\it Systematics-insensitive periodic signal search with K2}, {\bf accepted
for publication in ApJ}, arXiv:1505.07105.

\item
{\bf Ruth Angus} \& David Kipping, {\it Probabilistic inference of Basic
Stellar Parameters: Application to Flickering Stars}, in review, ApJ.

\item
{\bf Ruth Angus}, Suzanne Aigrain \& Dan Foreman-Mackey, {\it Stellar rotation
period inference with Gaussian processes}, Proceedings of the IAU XXIX, in
press.

\item
Andrew Vanderburg \& others, including {\bf Ruth Angus} 2015,
{\it A disintegrating minor planet transiting a white dwarf}, Nature, 526,
546.

\item
{\bf Ruth Angus}, Suzanne Aigrain, Daniel Foreman-Mackey \& Amy McQuillan
2015, {\it Calibrating gyrochronology using Kepler asteroseismic targets},
MNRAS, 450, 1787.

\item
Andrew Vanderburg \& others, including {\bf Ruth Angus} 2015,
{\it Characterizing K2 Planet Discoveries: A Super-Earth Transiting the
Bright K Dwarf HIP116454}, ApJ, 800, 59

\item
Parviainen, H. \& others, including {\bf Ruth Angus} 2014,
{\it Transiting exoplanets from the CoRoT space mission. XXV. CoRoT-27b: a
massive and dense planet on a short-period orbit} Astronomy \& Astrophysics
562, A140

\item
Aigrain, S., Alencar, S., {\bf Angus, R.}, Bouvier, J., Flaccomio, E., Gillen,
E., Guzik, J., Hebb, L., Hodgkin, S., McQuillan, A., Micela, G., Moraux, E.,
Parviainen, H., Randich, S., Reece, S., Roberts, S., Zwintz, K., 2013 {\it
Monitoring young associations and open clusters with Kepler in two-wheel mode}
arXiv:1309.0737

\item
Montet, Benjamin T., {\bf Angus, Ruth}, Barclay, Tom, Dawson, Rebekah, Fergus,
Rob, Foreman-Mackey, Dan, Harmeling, Stefan, Hirsch, Michael, Hogg, David W.,
Lang, Dustin, Schiminovich, David, Scholkopf, Bernhard, 2013, {\it Maximizing
Kepler science return per telemetered pixel: Searching the habitable zones of
the brightest stars}, arXiv:1309.0654

\item
Hogg, David W., {\bf Angus, Ruth}, Barclay, Tom, Dawson, Rebekah, Fergus, Rob,
Foreman-Mackey, Dan, Harmeling, Stefan, Hirsch, Michael, Lang, Dustin, Montet,
Benjamin T., Schiminovich, David, Schlkopf, Bernhard, 2013, {\it Maximizing
Kepler science return per telemetered pixel: Detailed models of the focal
plane in the two-wheel era}, arXiv:1309.0653

\item
Coe, M. J., {\bf Angus, R.}, Orosz, J. A., Udalski, A., 2013, {\it A detailed
study of the modulation of the optical light from Sk160/SMC X-1}, MNRAS 433,
746

    \end{list}
\fi

\end{document}
