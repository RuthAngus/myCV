% The formatting of this CV is based on @davidwhogg's layout.

\documentclass[12pt,letterpaper]{article}

\input{cvstyle}

\begin{document}\thispagestyle{empty}\sloppy\sloppypar\raggedbottom

\textbf{\Large Ruth Angus} \hfill
\textsf{\small RuthAngus@gmail.com} \\[0.5ex]
University of Oxford, Denys-Wilkinson Building, Keble Road, Oxford, OX1
3RH\\[0.5ex]

\cvheading{Education}
\begin{list}{}{\cvlist}
\item
D. Phil (Ph.D.) June 2016, Subdepartment of Astrophysics, University of
Oxford.
Advisor: Professor Suzanne Aigrain.
\item
Predoctoral fellowship 2015, Harvard-Smithsonian Center for Astrophysics.
Advisor: Professor John Asher Johnson.
\item
MPhys Physics with Astrophysics 2012, Department of Physics, University of
Southampton, UK. Advisor: Dr David Latham (Harvard-Smithsonian Center for
Astrophysics).
\end{list}

\cvheading{Academic Awards}
\begin{list}{}{\cvlist}
\item
Simons Fellowship (2016-2019).
\item
Predoctoral Fellowship, Harvard-Smithsonian Center for Astrophysics
(2014-2015).
\item
Leverhulme Trust funding award (2013-2014).
\item
Science and Technologies Facilities Council funding award (2012-2013).
\item
Highest score in third year Physics undergraduate studies, University of
Southampton (2011).
\item
Highest overall score in Physics undergraduate studies, University of
Southampton (2012).
\end{list}

\cvheading{Astrophysics review}
\begin{list}{}{\cvlist}
\item
Active referee for: Nature; The Astrophysical Journal;
The Astrophysical Journal, Letters and Monthly Notices of the Royal
Astronomical Society.
\end{list}

\cvheading{Principal investigator observing projects and observing experience}
\begin{list}{}{\cvlist}
\item
PI: {\it Searching for Super-Earths orbiting evolved stars} Planet Finder
Spectrograph on the Magellan Telescope, Las Campanas, Chile, 2 nights awarded,
2015.
\item
PI: {\it Optimising an observing strategy to search for planets orbiting
evolved stars} Planet Finder Spectrograph on the Magellan Telescope, Las
Campanas, Chile, 2 nights awarded, 2015.
\item
Observer: Sophie spectrograph, Observatoire d'Haute Provence, 6 nights,
September 2013.
\item
Observer: TRES spectrograph, Fred Lawrence Whipple Observatory, 5 nights,
November 2011.
\end{list}

\cvheading{Recent talks and tutorials}
\begin{list}{}{\cvlist}
\item
{\it Inferring the ages of Sun-like stars using photometric rotation periods},
Seminar, Max Planck Institute for Astronomy, Heidelberg, July 2016.
\item
{\it Exploring gyrochronology with LSST}, Contributed talk, Cool Stars 19,
Uppsala, Sweden, June 2016.
\item
{\it K2 asteroseismology}, Invited review talk, K2 special session, AAS 227,
Kissimee, Florida, January 2016.
\item
{\it Stellar ages from stellar rotation}, Dissertation talk, AAS 227,
Kissimee, Florida, January 2016.
\item
{\it Towards an age for every star: calibrating the age-rotation relations},
Astrophysics Colloquium, The Ohio State University, September 2015.
\item
{\it Stellar ages and stellar rotation}, invited seminar, Princeton
University, September 2015.
\item
{\it Probabilistic stellar rotation periods with Gaussian processes},
invited seminar, Institute for Advanced Study, September 2015.
\item
{\it Towards an age for every star: calibrating the age-rotation relations},
invited seminar, Canadian Institute for Theoretical Astrophysics,
University of Toronto, September 2015.
\item
{\it Stellar rotation, asteroseismology and gyrochronology}, invited seminar,
Harvard-Smithsonian Center for Astrophysics, August 2015.
\item
{\it Gyrochronology with K2 and TESS}, contributed talk, International
Astronomical Union meeting XXIX, Honolulu, HI, August 2015.
\item
{\it Probabilistic stellar rotation periods with Gaussian processes},
contributed talk, International Astronomical Union meeting XXIX, Honolulu, HI,
August 2015.
\item
{\it What K2 can do for gyrochronology}, invited seminar, NASA Ames, Mountain
View, CA, July 2015.
\item
{\it An introduction to Gaussian processes}, tutorial, Harvard-Smithsonian
Center for Astrophysics, July 2015.
\item
{\it The Systematics-Insensitive Periodogram for K2}, invited seminar,
Massachussetts Institute for Technology, June 2015.
\item
{\it The Systematics-Insensitive Periodogram for K2}, contributed talk,
Emerging researchers in exoplanets symposium, Penn State University, May 2015.
\item
{\it Gyrochronology and asteroseismology with Kepler and K2}, invited
seminar, Boston University, April 2015.
\item
{\it Calibrating gyrochronology}, contributed talk, American Astronomical
Society meeting, Seattle, WA, January 2015.
\item
{\it Calibrating gyrochronology using Kepler asteroseismic targets}, invited
seminar, University of Sheffield, UK, December 2014.
\item
{\it Calibrating gyrochronology using Kepler asteroseismic targets},
contributed talk, Planets across the HR diagram, Cambridge University, UK,
August 2014.
\item
{\it Calibrating gyrochronology using Kepler asteroseismic targets},
contributed talk, Cool Stars 18, Flagstaff, AZ, June 2014.
{\it Measuring stellar rotation periods using Gaussian processes},
contributed talk, Exoplanet statistics meeting, Carnegie-Mellon University,
Pittburgh, PA, June 2014.
\item
{\it Calibrating gyrochronology using Kepler asteroseismic targets}, invited
seminar, New York University, New York, March 2014.
\item
{\it Gyrochronology, stellar rotation and Gaussian processes}, invited
seminar, Harvard-Smithsonian Center for Astrophysics, February 2014.
\item
{\it Calibrating gyrochronology using Kepler asteroseismic targets}, Kepler
science conference, NASA Ames, Mountain View, CA, November 2013.
\end{list}

\cvheading{Writing, teaching and public outreach}
\begin{list}{}{\cvlist}
\item
{\it Exoplanets: distant worlds in our galaxy} public talks at Oxford
Prospects Summer School, Oxford, UK, July 2016.
\item
Local organiser: dotAstronomy conference, Oxford, UK, June 2016.
\item
Organiser: Cool Stars 19 hack day, Uppsala, Sweden, June 2016.
\item
Scientific expert: Channel 5 News, March 2015.
\item
Scientific expert: The Sky at Night (BBC television), November 2015.
\item
Teaching: Astrophysics labs demonstrator, University of Oxford, 2013-2014.
\item
Writer: BBC focus magazine book reviews (2014).
\item
Writer: Astrobites blog (2013-2015).
\item
{\it The end of the world is nigh!}, Public talk, Green Man festival, Brecon
Beacons, Wales, UK, August 2014.
\item
{\it The end of the world is nigh!}, Public talk, Wilderness festival, Brecon
Beacons, Wales, UK, August 2014.
\item
{\it Exoplanets and where to find them}, Public talk, The Idler magazine,
London, July, 2014.
\item
{\it Dark Sky Status}, Play, {\it Curious Directive} theater company,
Performer and scientific advisor, Latitude festival, UK, July, 2013.
\item
{\it Extrasolar planets: from science fiction to reality}, winner of the
Physics graduate outreach talk competition, University of Oxford, 2012.
\end{list}

\ifdefined\withpubs
    \cvheading{First Author Publications}
    \begin{list}{}{\cvlist}
    \input{pubs_first_author}
    \end{list}
\fi

\ifdefined\withpubs
    \cvheading{Co-authored Publications}
    \begin{list}{}{\cvlist}
    \input{pubs_coauthor}
    \end{list}
\fi

\end{document}
