% The formatting of this CV is based on @davidwhogg's layout.

\documentclass[12pt,letterpaper]{article}

\usepackage{color}
\usepackage{fancyhdr}
\usepackage{hyperref}
\usepackage{ifthen}

% Date formatting.
\usepackage[yyyymmdd]{datetime}
\renewcommand{\dateseparator}{-}

% Link formatting.
\definecolor{linkcolor}{rgb}{0,0,0.4}
\hypersetup{%
    colorlinks=true,        % false: boxed links; true: colored links
    linkcolor=linkcolor,    % color of internal links
    citecolor=linkcolor,    % color of links to bibliography
    filecolor=linkcolor,    % color of file links
    urlcolor=linkcolor      % color of external links
}

% Text formatting.
\newcommand{\foreign}[1]{\textit{#1}}
\newcommand{\etal}{\foreign{et~al.}}
\newcommand{\project}[1]{\textsl{#1}}
\definecolor{grey}{rgb}{0.5,0.5,0.5}
\newcommand{\deemph}[1]{\textcolor{grey}{\footnotesize{#1}}}

% literature links--use doi if you can
  \newcommand{\doi}[2]{\emph{\href{http://dx.doi.org/#1}{{#2}}}}
  \newcommand{\ads}[2]{\href{http://adsabs.harvard.edu/abs/#1}{{#2}}}
  \newcommand{\isbn}[1]{{\footnotesize(\textsc{isbn:}{#1})}}
  \newcommand{\arxiv}[1]{{\href{http://arxiv.org/abs/#1}{arXiv:{#1}}}}

% Section headings.
\newcommand{\cvheading}[1]{\addvspace{1ex}\pagebreak[2]\par\textbf{#1}\nopagebreak\vspace{-0.4em}}

% Set up the custom unordered list.
\newcounter{refpubnum}
\newcommand{\cvlist}{%
    \rightmargin=0in
    \leftmargin=0.15in
    \topsep=0ex
    \partopsep=0pt
    \itemsep=0.2ex
    \parsep=0pt
    \itemindent=-1.0\leftmargin
    \listparindent=0.0\leftmargin
    \settowidth{\labelsep}{~}
    \usecounter{refpubnum}
}

% Margins and spaces.
\raggedright
\setlength{\oddsidemargin}{0in}
\setlength{\topmargin}{0in}
\setlength{\headsep}{0.20in}
\setlength{\headheight}{0.25in}
\setlength{\textheight}{9.1in}
\addtolength{\topmargin}{-\headsep}
\addtolength{\topmargin}{-\headheight}
\setlength{\textwidth}{6.50in}
\setlength{\parindent}{0in}
\setlength{\parskip}{1ex}

% Headings and footings.
\renewcommand{\headrulewidth}{0pt}
\pagestyle{fancy}
\lhead{\deemph{Dr Ruth Angus}}
\chead{\deemph{Curriculum Vitae}}
\rhead{\deemph{\thepage}}
\cfoot{\deemph{Last updated: \today}}

% Journal names.
\newcommand{\aj}{AJ}
\newcommand{\apj}{ApJ}
\newcommand{\pasp}{PASP}
\newcommand{\mnras}{MNRAS}


\begin{document}\thispagestyle{empty}\sloppy\sloppypar\raggedbottom

\textbf{\Large Ruth Angus} \hfill
\textsf{\small RuthAngus@gmail.com} \\[0.5ex]
Simons postdoctoral junior fellow, Department of Astronomy, Columbia
University, 550 W 120th St, New York, NY 10027\\[0.5ex]

\cvheading{Education}
\begin{list}{}{\cvlist}
\item
D. Phil (Ph.D.) June 2016, Subdepartment of Astrophysics, University of
Oxford.
Advisor: Professor Suzanne Aigrain.
\item
Predoctoral fellowship 2015, Harvard-Smithsonian Center for Astrophysics.
Advisor: Professor John Asher Johnson.
\item
MPhys Physics with Astrophysics 2012, Department of Physics, University of
Southampton, UK. Advisor: Dr David Latham (Harvard-Smithsonian Center for
Astrophysics).
\end{list}

\cvheading{Academic Awards}
\begin{list}{}{\cvlist}
\item
Simons Fellowship (2016-2019).
\item
Predoctoral Fellowship, Harvard-Smithsonian Center for Astrophysics
(2014-2015).
\item
Leverhulme Trust funding award (2013-2014).
\item
Science and Technologies Facilities Council funding award (2012-2013).
\item
Highest score in third year Physics undergraduate studies, University of
Southampton (2011).
\item
Highest overall score in Physics undergraduate studies, University of
Southampton (2012).
\end{list}

% \cvheading{Astrophysics review}
% \begin{list}{}{\cvlist}
% \item
% Active referee for: Nature, The Astrophysical Journal,
% The Astrophysical Journal, Letters and Monthly Notices of the Royal
% Astronomical Society.
% \end{list}

\cvheading{Principal investigator observing projects}
\begin{list}{}{\cvlist}
\item
PI: {\it Radial velocity follow-up of Gaia wide-binary candidates in the
    Kepler field}, ModSpec spectrograph on the Hiltner 2.4m telescope, MDM
    observatory, Kitt Peak, AZ, 5 nights awarded 2017.
\item
    PI: {\it Searching for Super-Earths orbiting evolved stars} Planet Finder
Spectrograph on the Magellan Telescope, Las Campanas, Chile, 2 nights awarded,
    2015.
\item
PI: {\it Optimising an observing strategy to search for planets orbiting
evolved stars} Planet Finder Spectrograph on the Magellan Telescope, Las
    Campanas, Chile, 2 nights awarded, 2015.
% \item
% Observer: Sophie spectrograph, Observatoire d'Haute Provence, 6 nights,
% September 2013.
% \item
% Observer: TRES spectrograph, Fred Lawrence Whipple Observatory, 5 nights,
% November 2011.
\end{list}

% \cvheading{Recent talks and tutorials}
% \begin{list}{}{\cvlist}
% \item
% {\it Inferring the ages of Sun-like stars using photometric rotation periods},
% Seminar, Max Planck Institute for Astronomy, Heidelberg, July 2016.
% \item
% {\it Exploring gyrochronology with LSST}, Contributed talk, Cool Stars 19,
% Uppsala, Sweden, June 2016.
% \item
% {\it K2 asteroseismology}, Invited review talk, K2 special session, AAS 227,
% Kissimee, Florida, January 2016.
% \item
% {\it Stellar ages from stellar rotation}, Dissertation talk, AAS 227,
% Kissimee, Florida, January 2016.
% \item
% {\it Towards an age for every star: calibrating the age-rotation relations},
% Astrophysics Colloquium, The Ohio State University, September 2015.
% \item
% {\it Stellar ages and stellar rotation}, invited seminar, Princeton
% University, September 2015.
% \item
% {\it Probabilistic stellar rotation periods with Gaussian processes},
% invited seminar, Institute for Advanced Study, September 2015.
% \item
% {\it Towards an age for every star: calibrating the age-rotation relations},
% invited seminar, Canadian Institute for Theoretical Astrophysics,
% University of Toronto, September 2015.
% \item
% {\it Stellar rotation, asteroseismology and gyrochronology}, invited seminar,
% Harvard-Smithsonian Center for Astrophysics, August 2015.
% \item
% {\it Gyrochronology with K2 and TESS}, contributed talk, International
% Astronomical Union meeting XXIX, Honolulu, HI, August 2015.
% \item
% {\it Probabilistic stellar rotation periods with Gaussian processes},
% contributed talk, International Astronomical Union meeting XXIX, Honolulu, HI,
% August 2015.
% \item
% {\it What K2 can do for gyrochronology}, invited seminar, NASA Ames, Mountain
% View, CA, July 2015.
% \item
% {\it An introduction to Gaussian processes}, tutorial, Harvard-Smithsonian
% Center for Astrophysics, July 2015.
% \item
% {\it The Systematics-Insensitive Periodogram for K2}, invited seminar,
% Massachussetts Institute for Technology, June 2015.
% \item
% {\it The Systematics-Insensitive Periodogram for K2}, contributed talk,
% Emerging researchers in exoplanets symposium, Penn State University, May 2015.
% \item
% {\it Gyrochronology and asteroseismology with Kepler and K2}, invited
% seminar, Boston University, April 2015.
% \item
% {\it Calibrating gyrochronology}, contributed talk, American Astronomical
% Society meeting, Seattle, WA, January 2015.
% \item
% {\it Calibrating gyrochronology using Kepler asteroseismic targets}, invited
% seminar, University of Sheffield, UK, December 2014.
% \item
% {\it Calibrating gyrochronology using Kepler asteroseismic targets},
% contributed talk, Planets across the HR diagram, Cambridge University, UK,
% August 2014.
% \item
% {\it Calibrating gyrochronology using Kepler asteroseismic targets},
% contributed talk, Cool Stars 18, Flagstaff, AZ, June 2014.
% {\it Measuring stellar rotation periods using Gaussian processes},
% contributed talk, Exoplanet statistics meeting, Carnegie-Mellon University,
% Pittburgh, PA, June 2014.
% \item
% {\it Calibrating gyrochronology using Kepler asteroseismic targets}, invited
% seminar, New York University, New York, March 2014.
% \item
% {\it Gyrochronology, stellar rotation and Gaussian processes}, invited
% seminar, Harvard-Smithsonian Center for Astrophysics, February 2014.
% \item
% {\it Calibrating gyrochronology using Kepler asteroseismic targets}, Kepler
% science conference, NASA Ames, Mountain View, CA, November 2013.
% \end{list}

% \cvheading{Writing, teaching and public outreach}
% \begin{list}{}{\cvlist}
% \item
% {\it Exoplanets: distant worlds in our galaxy} public talks at Oxford
% Prospects Summer School, Oxford, UK, July 2016.
% \item
% Local organiser: dotAstronomy conference, Oxford, UK, June 2016.
% \item
% Organiser: Cool Stars 19 hack day, Uppsala, Sweden, June 2016.
% \item
% Scientific expert: Channel 5 News, March 2015.
% \item
% Scientific expert: The Sky at Night (BBC television), November 2015.
% \item
% Teaching: Astrophysics labs demonstrator, University of Oxford, 2013-2014.
% \item
% Writer: BBC focus magazine book reviews (2014).
% \item
% Writer: Astrobites blog (2013-2015).
% \item
% {\it The end of the world is nigh!}, Public talk, Green Man festival, Brecon
% Beacons, Wales, UK, August 2014.
% \item
% {\it The end of the world is nigh!}, Public talk, Wilderness festival, Brecon
% Beacons, Wales, UK, August 2014.
% \item
% {\it Exoplanets and where to find them}, Public talk, The Idler magazine,
% London, July, 2014.
% \item
% {\it Dark Sky Status}, Play, {\it Curious Directive} theater company,
% Performer and scientific advisor, Latitude festival, UK, July, 2013.
% \item
% {\it Extrasolar planets: from science fiction to reality}, winner of the
% Physics graduate outreach talk competition, University of Oxford, 2012.
% \end{list}

\cvheading{First Author Publications}
\begin{list}{}{\cvlist}
\item
    {\bf Angus, R.} \& Kipping, D. {\it Probabilistic Inference of Basic
    Stellar Parameters: Application to Flickering Stars, 2016, ApJ Letters,
    823, 9.}
\item
    {\bf Angus, R.}, Foreman-Mackey, D., Johnson, A., J., {\it
    Systematics-insensitive Periodic Signal Search with K2}, 2016, ApJ, 818,
    109
\item
    {\bf Angus, R.}, Aigrain, S., Foreman-Mackey, D., McQuillan, A., {\it
    Calibrating Gyrochronology using Kepler Asteroseismic Targets}, 2015,
    MNRAS, 225, 112.
\end{list}

\cvheading{Co-authored Publications}
\begin{list}{}{\cvlist}
\item
Vanderburg, Andrew, Johnson, John Asher, Rappaport, Saul, Bieryla, Allyson,
    Irwin, Jonathan, Lewis, John Arban, Kipping, David, Brown, Warren R.,
    Dufour, Patrick, Ciardi, David R., {\bf Angus, Ruth}, Schaefer, Laura,
    Latham, David W., Charbonneau, David, Beichman, Charles, Eastman, Jason,
    McCrady, Nate, Wittenmyer, Robert A., Wright, Jason T.
    {\it A disintegrating minor planet transiting a white dwarf}, 2015,
    Nature, 526, 7574, 546.
\item
Vanderburg, Andrew, Montet, Benjamin T., Johnson, John Asher, Buchhave, Lars
    A., Zeng, Li, Pepe, Francesco, Collier Cameron, Andrew, Latham, David W.,
    Molinari, Emilio, Udry, Stéphane, Lovis, Christophe, Matthews, Jaymie M.,
    Cameron, Chris, Law, Nicholas, Bowler, Brendan P., {\bf Angus, Ruth},
    Baranec, Christoph, Bieryla, Allyson, Boschin, Walter, Charbonneau, David,
    Cosentino, Rosario, Dumusque, Xavier, Figueira, Pedro, Guenther, David B.,
    Harutyunyan, Avet, Hellier, Coel, Kuschnig, Rainer, Lopez-Morales,
    Mercedes, Mayor, Michel, Micela, Giusi, Moffat, Anthony F. J., Pedani,
    Marco, Phillips, David F., Piotto, Giampaolo, Pollacco, Don, Queloz,
    Didier, Rice, Ken, Riddle, Reed, Rowe, Jason F., Rucinski, Slavek M.,
    Sasselov, Dimitar, Ségransan, Damien, Sozzetti, Alessandro, Szentgyorgyi,
    Andrew, Watson, Chris, Weiss, Werner W.
    {\it Characterizing K2 Planet Discoveries: A Super-Earth Transiting the
    Bright K Dwarf HIP 116454}, 2015, ApJ, 800, 59.
\item
Parviainen, H., Gandolfi, D., Deleuil, M., Moutou, C., Deeg, H. J.,
    Ferraz-Mello, S., Samuel, B., Csizmadia, Sz., Pasternacki, T., Wuchterl,
    G., Havel, M., Fridlund, M., {\bf Angus, R.}, Tingley, B., Grziwa, S.,
    Korth, J., Aigrain, S., Almenara, J. M., Alonso, R., Baglin, A., Barros,
    S. C., C., Bordé, P., Bouchy, F., Cabrera, J., Díaz, R. F., Dvorak, R.,
    Erikson, A., Guillot, T., Hatzes, A., Hébrard, G., Mazeh, T., Montagnier,
    G., Ofir, A., Ollivier, M., Pätzold, M., Rauer, H., Rouan, D., Santerne,
    A., Schneider, J.
    {\it Transiting exoplanets from the CoRoT space mission. XXV. CoRoT-27b: a
    massive and dense planet on a short-period orbit}, 2014, Astronomy \&
    Astrophysics, 562, 140.
\item
    Coe, M. J., {\bf Angus, R.}, Orosz, J. A., Udalski, A. {\it A detailed
    study of the modulation of the optical light from Sk160/SMC X-1}, 2013,
    MNRAS, 433, 746.
\end{list}

\cvheading{Non-refereed Publications}
\begin{list}{}{\cvlist}
\item
    {\bf Angus, R.}, Morton, T., Aigrain, S. \& Foreman-Mackey, D., {\it
    Inferring stellar rotation periods using Gaussian processes}, MNRAS
    submitted (in review).
\item
    Foreman-Mackey, D., Agol, E., {\bf Angus, R.}, Ambikasaran, S., {\it Fast
    and scalable Gaussian process modeling with applications to astronomical
    time series}, ApJ submitted (in review), https://arxiv.org/abs/1703.09710
\item
    Najita, J., Willman, B., Finkbeiner, D., P., \& others, including {\bf
    Angus, R.}, {\it Maximizing Science in the Era of LSST: A Community-Based
    Study of Needed US Capabilities}, 2016, ArXiv only,
    https://arxiv.org/abs/1610.01661
\item
    Hawley, S. L., {\bf Angus, R.}, Buzasi, D., Davenport, J., R., A.,
    Giampapa, M., Kashyap, V., Meibom, S., {\it Maximizing Science in the Era
    of LSST, Stars Study Group Report: Rotation and Magnetic Activity in the
    Galactic Field Population and in Open Star Clusters}, 2016, ArXiv only,
    https://arxiv.org/abs/1607.04302
\item
    Aigrain, S., Alencar, S., {\bf Angus, R.}, Bouvier, J., Flaccomio, E.,
    Gillen, E., Guzik, J., Hebb, L., Hodgkin, S., McQuillan, A., Micela, G.,
    Moraux, E., Parviainen, H., Randich, S., Reece, S., Roberts, S.,
    Zwintz, K., {\it Monitoring young associations and open clusters with
    Kepler in two-wheel mode}, 2013, ArXiv only,
    https://arxiv.org/abs/1309.0737
\item
    Montet, Benjamin T., {\bf Angus, Ruth}, Barclay, Tom, Dawson, Rebekah,
    Fergus, Rob, Foreman-Mackey, Dan, Harmeling, Stefan, Hirsch, Michael,
    Hogg, David W., Lang, Dustin, Schiminovich, David, Scholkopf, Bernhard
    {\it Maximizing Kepler science return per telemetered pixel: Searching the
    habitable zones of the brightest stars}, 2013, ArXiv only,
    https://arxiv.org/abs/1309.0654
\item
    Hogg, David W., {\bf Angus, Ruth}, Barclay, Tom, Dawson, Rebekah, Fergus,
    Rob, Foreman-Mackey, Dan, Harmeling, Stefan, Hirsch, Michael, Lang,
    Dustin, Montet, Benjamin T., Schiminovich, David, Schölkopf, Bernhard {\it
    Maximizing Kepler science return per telemetered pixel: Detailed models of
    the focal plane in the two-wheel era}, 2013, ArXiv only,
    https://arxiv.org/abs/1309.0653
\end{list}

% \ifdefined\withpubs
%     \cvheading{First Author Publications}
%     \begin{list}{}{\cvlist}
%     \input{pubs_first_author}
%     \end{list}
% \fi

% \ifdefined\withpubs
%     \cvheading{Co-authored Publications}
%     \begin{list}{}{\cvlist}
%     \input{pubs_coauthor}
%     \end{list}
% \fi

\end{document}
