% The formatting of this CV is based on @davidwhogg's layout.

\documentclass[12pt,letterpaper]{article}

\input{cvstyle}

\begin{document}\thispagestyle{empty}\sloppy\sloppypar\raggedbottom

\textbf{\Large Ruth Angus} \hfill
\textsf{\small rangus@amnh.org} \\[0.5ex]
Assistant Curator of Astrophysics at the American Museum of Natural History,
Associate Research Scientist at the Center for Computational Astrophysics at
the Flatiron Institute,
and Assistant Adjunct Professor at Columbia University.
\\[0.5ex]
% In September 2018 I will become an Assistant Curator at the American Museum of
% Natural History, Associate Research Scientist at the Center for Computational
% Astrophysics of the Flatiron Institute and Adjuct Professor at Columbia
% University.\\[0.5ex]

\cvheading{Academic Appointments}
\begin{list}{}{\cvlist}
\item
    October 2018 - Present: Assistant Curator of Astrophysics, American Museum
    of Natural History, Division of Physical Sciences, Central Park West \&
    79th St, New York, NY 10024.
\item
    October 2018 - Present: Associate Research Scientist, Flatiron Institute,
    Center for Computational Astrophysics, 162 5th Avenue, New York, NY,
    10010.
\item
    October 2018 - Present: Assistant Adjunct Professor of Astronomy,
    Columbia University, Department of Astronomy, Pupin Hall, New York, NY,
    10027.
\item
    October 2016 - October 2018: Simons Junior Postdoctoral Fellow,
    Department of Astronomy, Columbia University, 550 W 120th St, New York,
    NY 10027.
\end{list}

\cvheading{Education}
\begin{list}{}{\cvlist}
\item
D. Phil (Ph.D.) June 2016, Subdepartment of Astrophysics, University of
Oxford.
Advisor: Professor Suzanne Aigrain.
\item
Predoctoral fellowship 2015, Harvard-Smithsonian Center for Astrophysics.
Advisor: Professor John Asher Johnson.
\item
MPhys Physics with Astrophysics 2012, Department of Physics, University of
Southampton, UK. Advisor: Dr David Latham (Harvard-Smithsonian Center for
Astrophysics).
\end{list}

\cvheading{Grants and Awards}
\begin{list}{}{\cvlist}
\item
    K2 guest observer grant awarded for `A novel approach to age analysis for
    Kepler M dwarfs', Co-Investigator, July 2018.
\item
    NASA Astrophyics Data Analysis Program (ADAP) grant awarded for
    `Stellar Rotation Periods from the K2 Spacecraft', Co-Investigator,
    September 2017
\item
Simons Fellowship (2016-2019).
\item
Predoctoral Fellowship, Harvard-Smithsonian Center for Astrophysics
(2014-2015).
\item
Leverhulme Trust award (2013-2014).
\item
Science and Technologies Facilities Council award (2012-2013).
\item
Highest overall score in Physics undergraduate studies, University of
Southampton (2012).
\end{list}

\cvheading{Astronomical Community Service}
\begin{list}{}{\cvlist}
\item
    Co-organizer of the TESS workshop at StSci, Baltimore, MD, February 2019.
    Review panel member for NASA ADAP grants, 2018.
    Review panel member for Neural Information Processing Systems workshop
    `Machine learning for the physical sciences', 2017.
    Referee for Nature, The Astrophysical
    Journal, The Astrophysical Journal Letters Monthly Notices of the
    Royal Astronomical Society and The Journal of Open Source Software.
\end{list}

% \cvheading{Principal investigator observing projects}
% \begin{list}{}{\cvlist}
% \item
% PI: {\it Radial velocity follow-up of Gaia wide-binary candidates in the
%     Kepler field}, ModSpec spectrograph on the Hiltner 2.4m telescope, MDM
%     observatory, Kitt Peak, AZ, 5 nights awarded 2017.
% \item
%     PI: {\it Searching for Super-Earths orbiting evolved stars} Planet Finder
% Spectrograph on the Magellan Telescope, Las Campanas, Chile, 2 nights awarded,
%     2015.
% \item
% PI: {\it Optimising an observing strategy to search for planets orbiting
% evolved stars} Planet Finder Spectrograph on the Magellan Telescope, Las
%     Campanas, Chile, 2 nights awarded, 2015.
% % \item
% % Observer: Sophie spectrograph, Observatoire d'Haute Provence, 6 nights,
% % September 2013.
% % \item
% % Observer: TRES spectrograph, Fred Lawrence Whipple Observatory, 5 nights,
% % November 2011.
% \end{list}

\cvheading{Recent talks and tutorials}
\begin{list}{}{\cvlist}
\item
    {\it Rotation and activity in clusters}, Invited talk, the 10th
    anniversary Kepler Science Meeting, LA, March 2019.
\item
    {\it Planetary systems across time and space}, Colloquium, Stony Brook
    University, Stony Brook NY, February 2019.
\item
    {\it Stardate: a new tool for measuring stellar ages}, Contributed talk,
    American Astronomical Society winter meeting, Seattle, WA, January 2019.
\item
    {\it An introduction to Bayesian Inference}, tutorial, Astro hack
    week, Lorentz Centre, Leiden, Netherlands, August 2018.
\item
    {\it The ages of exoplanet hosts}, contributed talk, Cool stars 20
    conference, Boston, July 2018.
\item
    {\it Inferring stellar ages}, invited talk, BayesComp conference,
    Barcelona, April 2018.
\item
    {\it Planetary systems across space and time}, Astronomy Colloquium,
    University of Delaware, March 2018.
\item{\it An introduction to Gaussian Processes}, K2 clusters workshop,
    Boston University, January 2018.
\end{list}

% {\it Inferring the ages of Sun-like stars using photometric rotation periods},
% Seminar, Max Planck Institute for Astronomy, Heidelberg, July 2016.
% \item
% {\it Exploring gyrochronology with LSST}, Contributed talk, Cool Stars 19,
% Uppsala, Sweden, June 2016.
% \item
% {\it K2 asteroseismology}, Invited review talk, K2 special session, AAS 227,
% Kissimee, Florida, January 2016.
% \item
% {\it Stellar ages from stellar rotation}, Dissertation talk, AAS 227,
% Kissimee, Florida, January 2016.
% \item
% {\it Towards an age for every star: calibrating the age-rotation relations},
% Astrophysics Colloquium, The Ohio State University, September 2015.
% \item
% {\it Stellar ages and stellar rotation}, invited seminar, Princeton
% University, September 2015.
% \item
% {\it Probabilistic stellar rotation periods with Gaussian processes},
% invited seminar, Institute for Advanced Study, September 2015.
% \item
% {\it Towards an age for every star: calibrating the age-rotation relations},
% invited seminar, Canadian Institute for Theoretical Astrophysics,
% University of Toronto, September 2015.
% \item
% {\it Stellar rotation, asteroseismology and gyrochronology}, invited seminar,
% Harvard-Smithsonian Center for Astrophysics, August 2015.
% \item
% {\it Gyrochronology with K2 and TESS}, contributed talk, International
% Astronomical Union meeting XXIX, Honolulu, HI, August 2015.
% \item
% {\it Probabilistic stellar rotation periods with Gaussian processes},
% contributed talk, International Astronomical Union meeting XXIX, Honolulu, HI,
% August 2015.
% \item
% {\it What K2 can do for gyrochronology}, invited seminar, NASA Ames, Mountain
% View, CA, July 2015.
% \item
% {\it An introduction to Gaussian processes}, tutorial, Harvard-Smithsonian
% Center for Astrophysics, July 2015.
% \item
% {\it The Systematics-Insensitive Periodogram for K2}, invited seminar,
% Massachussetts Institute for Technology, June 2015.
% \item
% {\it The Systematics-Insensitive Periodogram for K2}, contributed talk,
% Emerging researchers in exoplanets symposium, Penn State University, May 2015.
% \item
% {\it Gyrochronology and asteroseismology with Kepler and K2}, invited
% seminar, Boston University, April 2015.
% \item
% {\it Calibrating gyrochronology}, contributed talk, American Astronomical
% Society meeting, Seattle, WA, January 2015.
% \item
% {\it Calibrating gyrochronology using Kepler asteroseismic targets}, invited
% seminar, University of Sheffield, UK, December 2014.
% \item
% {\it Calibrating gyrochronology using Kepler asteroseismic targets},
% contributed talk, Planets across the HR diagram, Cambridge University, UK,
% August 2014.
% \item
% {\it Calibrating gyrochronology using Kepler asteroseismic targets},
% contributed talk, Cool Stars 18, Flagstaff, AZ, June 2014.
% {\it Measuring stellar rotation periods using Gaussian processes},
% contributed talk, Exoplanet statistics meeting, Carnegie-Mellon University,
% Pittburgh, PA, June 2014.
% \item
% {\it Calibrating gyrochronology using Kepler asteroseismic targets}, invited
% seminar, New York University, New York, March 2014.
% \item
% {\it Gyrochronology, stellar rotation and Gaussian processes}, invited
% seminar, Harvard-Smithsonian Center for Astrophysics, February 2014.
% \item
% {\it Calibrating gyrochronology using Kepler asteroseismic targets}, Kepler
% science conference, NASA Ames, Mountain View, CA, November 2013.
% \end{list}

\cvheading{Recent public outreach and engagement}
\begin{list}{}{\cvlist}
\item
    {\it The time axis of astronomy}, Simon Foundation Public Lecture,
    New York, NY, February 2019.
\item Keynote speaker, Science research symposium, Tenafly High School,
    May 2018.
\item
    {\it Planetary systems across time and space}, Greenwich Astronomical
    Society, Greenwich, CT, April 2018.
\item
    {\it NPR All things considered}, Scientific expert, April 2018.
\item
    {\it Bad science in movies}, public talk at Science Exclamation Point,
    Caveat, NY, February 2018.
\item{\it Journey to an exoplanet}, immersive performance at Caveat, NY,
    December 2017.
\item{\it Undead spacecraft}, public talk at Astronomy on Tap, the Way
    Station, Brooklyn, NY, October 2017.
% \item{\it Exoplanets: distant worlds in our galaxy} public talks at Oxford
% Prospects Summer School, Oxford, UK, July 2016.
% % \item
% % Local organiser: dotAstronomy conference, Oxford, UK, June 2016.
% % \item
% % Organiser: Cool Stars 19 hack day, Uppsala, Sweden, June 2016.
\item
Scientific expert: Channel 5 News, March 2015.
\item
Scientific expert: The Sky at Night (BBC television), November 2015.
% % \item
% % Teaching: Astrophysics labs demonstrator, University of Oxford, 2013-2014.
\item
Writer: BBC focus magazine book reviews (2014).
% \item
% Writer: Astrobites blog (2013-2015).
\item
{\it The end of the world is nigh!}, Public talk, Green Man festival, Brecon
Beacons, Wales, UK, August 2014.
\item
{\it The end of the world is nigh!}, Public talk, Wilderness festival, Brecon
Beacons, Wales, UK, August 2014.
\item
{\it Exoplanets and where to find them}, Public talk, The Idler magazine,
London, July, 2014.
\item
{\it Dark Sky Status}, Play devised by {\it Curious Directive} theater
    company, Performer and scientific advisor, Latitude festival, UK, July,
    2013.
\item
{\it Extrasolar planets: from science fiction to reality}, winner of the
Physics graduate outreach talk competition, University of Oxford, 2012.
\end{list}


\cvheading{First Author Publications}
\begin{list}{}{\cvlist}
\item
    {\bf Angus, R.}, Morton, T., Aigrain, S., Foreman-Mackey, D., Rajpaul, V.,
    {\it Inferring probabilistic stellar rotation periods using Gaussian
    processes}, 2017, Monthly Notices of the Royal Astronomical Society
    474, 2094.
\item
    {\bf Angus, R.} \& Kipping, D. {\it Probabilistic Inference of Basic
    Stellar Parameters: Application to Flickering Stars}, 2016, The
    Astrophysical Journal Letters, 823, 9.
\item
    {\bf Angus, R.}, Foreman-Mackey, D., Johnson, A., J., {\it
    Systematics-insensitive Periodic Signal Search with K2}, 2016, The
    Astrophysical Journal, 818, 109.
\item
    {\bf Angus, R.}, Aigrain, S., Foreman-Mackey, D., McQuillan, A., {\it
    Calibrating Gyrochronology using Kepler Asteroseismic Targets}, 2015,
    Monthly Notices of the Royal Astronomical Society, 225, 112.
\end{list}

\cvheading{Co-authored Publications}
\begin{list}{}{\cvlist}
\item
    Morris, B. \& others including {\bf Angus, R.},
    {\it The Solar Benchmark: Rotational Modulation of the Sun Reconstructed
    from Archival Sunspot Records}
\item
    Ness, M. K. \& others including {\bf Angus, R.},
    {\it Inference of stellar parameters from brightness variations},
    The Astrophysical Journal, 866, 15
\item
    Foreman-Mackey, D., Agol, E., Ambikasaran, S., \& {\bf Angus, R.}, {\it
    Fast and scalable Gaussian process modeling with applications to
    astronomical time series}, The Astrophysical Journal, 154, 220.
\item
    Vanderburg, A., \& others including {\bf Angus, R.},
    {\it A disintegrating minor planet transiting a white dwarf}, 2015,
    Nature, 526, 7574, 546.
\item
Vanderburg, A., \& others including {\bf Angus, R.},
    {\it Characterizing K2 Planet Discoveries: A Super-Earth Transiting the
    Bright K Dwarf HIP 116454}, 2015, The Astrophysical Journal, 800, 59.
\item
Parviainen, H., \& others including {\bf Angus, R.},
    {\it Transiting exoplanets from the CoRoT space mission. XXV. CoRoT-27b: a
    massive and dense planet on a short-period orbit}, 2014, Astronomy \&
    Astrophysics, 562, 140.
\item
    Coe, M. J., {\bf Angus, R.}, Orosz, J. A., Udalski, A. {\it A detailed
    study of the modulation of the optical light from Sk160/SMC X-1}, 2013,
    Monthly Notices of the Royal Astronomical Society, 433, 746.
\end{list}

\cvheading{Non-refereed Publications}
\begin{list}{}{\cvlist}
\item
    LSST Science Collaboration \& others, including {\bf Angus, R.},
    {\it Science-Driven Optimization of the LSST Observing Strategy}, 2017,
    arXiv:1708.04058
\item
    Najita, J., \& others, including {\bf Angus, R.}, {\it Maximizing Science
    in the Era of LSST: A Community-Based Study of Needed US Capabilities},
    2016, arXiv:1610.01661
\item
    Hawley, S. L., {\bf Angus, R.}, Buzasi, D., Davenport, J., R., A.,
    Giampapa, M., Kashyap, V., Meibom, S., {\it Maximizing Science in the Era
    of LSST, Stars Study Group Report: Rotation and Magnetic Activity in the
    Galactic Field Population and in Open Star Clusters}, 2016,
    arxiv:1607.04302
\item
    Aigrain, S., \& others including {\bf Angus, R.},
    {\it Monitoring young associations and open clusters with
    Kepler in two-wheel mode}, 2013, arxiv:1309.0737
\item
    Montet, B. T., \& others including {\bf Angus, R.},
    {\it Maximizing Kepler science return per telemetered pixel: Searching the
    habitable zones of the brightest stars}, 2013, arxiv:1309.0654
\item
    Hogg, D., W., \& others including {\bf Angus, R.},
    {\it Maximizing Kepler science return per telemetered pixel: Detailed
    models of the focal plane in the two-wheel era}, 2013, arxiv:1309.0653
\end{list}

% \ifdefined\withpubs
%     \cvheading{First Author Publications}
%     \begin{list}{}{\cvlist}
%     \input{pubs_first_author}
%     \end{list}
% \fi

% \ifdefined\withpubs
%     \cvheading{Co-authored Publications}
%     \begin{list}{}{\cvlist}
%     \input{pubs_coauthor}
%     \end{list}
% \fi

\end{document}
